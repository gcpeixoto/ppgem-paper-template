% File: main.tex
% Date: 2015, April
% Authors: OLIVEIRA, G.P., LIMA, L.M.R.
% Description: Latex template file for seminar short papers presented at PPG-EM/UERJ.
% Version: 1.0
% Available online on: www.gesar.uerj.br

%% PREAMBLE
\documentclass[a4paper,links,date]{ppgem} 

%% TITLE
\Title{Preparing your abstract to be presented at PPG-EM's seminars: a first tutorial}

%% AUTHORS
\AuthorName{Oliveira, G.P., Matos, L.M.R.}
\CorrEmail{gustavo.oliveira@uerj.br}
\AdvisorName{Norberto Mangiavacchi}
\CoAdvisorName{} 

%% INSTITUTION
\InstA{State University of Rio de Janeiro} 
\InstB{} 

\begin{document}
\thispagestyle{plain}
\makeheader

\begin{multicols}{2}

%% KEYWORDS
\begin{keywords}
Scientific writing, paper writing, {\LaTeX} typesetting, standardisation.
\end{keywords}

%% BODY
\section{Introduction}

This short paper is intended to introduce a self-explained tutorial on how to prepare abstract texts to be presented in the form of internal seminars at Graduate Program in Mechanical Engineering (now on PPG-EM), from State University of Rio de Janeiro (UERJ). In order to suggest a standard formatting for better organization and registration at PPG-EM as well as to help incoming students to be acquainted with the {\LaTeX} typesetting, this paper also dismembers into a few goals, such as: i) to work as an introductory text for training in scientific writing among the PPG-EM's students and seminar attendees; ii) to strengthen the practical use of English language over the academic environment; iii) to provide guidelines to outline the first versions of those research issues that will may be turned into extended abstracts and/or conference papers and iv) to enhance the PPG-EM's academic competitiveness worldwide. 

\section{Text elements and organization}

Basically, your paper should have five major parts: i) Introduction; ii) Methodology; iii) Results; iv) Discussion and v) Conclusion, although the parts iv) e v) may be combined into a unique section. 

You are free to set out the title of your paper provided that you have good reasons to support your choice. It should be totally capitalised. All the sections and subsections should have only the first letter capitalised, except when a proper noun is required. The following examples could be used for titles:
\begin{itemize}
\item  \emph{ROBUST METHODS TO CALCULATE ERROR ESTIMATES IN DIV-CURL FORMULATIONS} \\
\item  \emph{EVALUATION OF DISTURBANCE MAGNITUDES FORMED FROM PULSATING WAVE SOURCES} \\
\item  \emph{TRAVELING THROUGH CONTINUUM MECHANICS: SHOULD WE USE BOOKS BY TRUESDELL, GURTIN, SPENCER, OR MASE?} 
\end{itemize}
whereas
\begin{itemize}
\item  \emph{Supercritical flows for $ 10 < Fr < 100 $} \\
\item  \emph{Physicochemistry of gold-copper nanoparticules} \\
\item  \emph{Action of the gradient vector: relation between growth rate and temperature scattering;} 
\end{itemize}
could be applied to sections or subsections. 

\section{Model and data presentation }

This section explains how to insert equations, figures and tables into your texr

\subsection{Equations}

Equations can be added to your text through the usual {\LaTeX}  environments to have a uniquely labelled equation like 
\begin{equation}
\label{eq:1}
  \frac{L}{A}\frac{dW}{dt}=\rho_0\beta g\oint Tdz-f\frac{L}{D}\frac{W^2}{2\rho_0A^2},
\end{equation}
and multi-line labelled equations like
\begin{subequations}
\label{eq:2}
\begin{eqnarray}
\frac{\D T}{\D t}+\frac{W}{A\rho_0}\frac{\D T}{\D s}&=&\frac{4q}{D\rho_0c_p} \label{eq:2a} \\
\frac{\D T}{\D t}+\frac{W}{A\rho_0}\frac{\D T}{\D s}&=&-\frac{4U(T-T_s)}{D\rho_0c_p}\\
\frac{\D T}{\D t}+\frac{W}{A\rho_0}\frac{\D T}{\D s}&=& 0 \label{eq:2c}
 \end{eqnarray}
\end{subequations}
or like
\begin{eqnarray}
\label{eq:3}
  f &=& 8\left[\left(\frac{8}{\Re}\right)^{12} + (A+B^{-1,5})\right]^{1/12} \nonumber \\
  A &=& \left\{-2,457\ln\left[\left(\frac{7}{\Re}\right)^{0,9} + \frac{0,27e}{D}\right]\right\}^{16} \\
  B &=& \left(\frac{37530}{\Re}\right)^{16}\nonumber
\end{eqnarray}
Equation \ref{eq:1} is the way how you should refer to an equation at the beginning of a statement. Equations (\ref{eq:2a}-\ref{eq:2c}) is the second way. Otherwise, if you need refer to another equation, you should write Eq.~(\ref{eq:2a}-\ref{eq:2c}) or just Eq.~\ref{eq:3}

\subsection{Figures}

Figures are added to your paper by calling 
\begin{verbatim}
\begin{figure}\includegraphics[...] 
(...)
 \end{figure}
\end{verbatim}
so that
\begin{figure}
%\begin{center}
\includegraphics[width=.6\columnwidth,angle=-90,origin=c]{figs/example}
%\caption{Mass flow with time.}
\label{mass_flow}
%\end{center}
\end{figure}

\subsection{Tables}

Tables like \ref{tbl:1} can also be inserted into your text.
\begin{center}
  \begin{tabular}{ccc}
    \hline
    & lower bound & upper bound\\
    \hline
    \citet{ambrosini2004} & 285 W & 480 W\\
    present model & 390 W & 707 W\\
    \hline\\
  \end{tabular}
  \tabcaption{Stability thresholds using Churchill's friction correlation, with external fluid temperature of 30ºC.}
  \label{tbl:1}
\end{center}

\subsection{Citations}

To cite other authors or references, use the textual and parenthetical commands provided by \verb|natbib| package 
\verb|\cite{ref1}| or \verb|\citep{ref1}|. Add your references to \verb|refs.bib| and compile by using \verb|bibtex|.
The usual bib entries are available. This paper's bibliography, for instance, is formed by: a M.Sc. thesis \citep{rabellomsc2007}, 
a tech report \cite{amarante2001}, a book \citep{batchelor1994}, an inproceedings \cite{lima2009}, a Ph.D. thesis 
\citep{loureirophd2008} and a misc \citep{mangiavacchi2000}.

\section{Conclusions}

Here, you will end up your text. In order to reduce it, we encourage you to summarize the main results by using an itemized list.
\begin{itemize}
\item the described model provided good steady state predictions;
\item transient predictions are considerably sensible to numerical parameters;
\item friction factor correlation plays an important role in such models;
\item 3D simulations may reveal important flow characteristics of NCLs.
\end{itemize}

\section{Acknowledgments}

This section is optional.

%% REFERENCES
\bibliographystyle{plainnat}
\bibliography{refs}

\end{multicols}
\end{document}
